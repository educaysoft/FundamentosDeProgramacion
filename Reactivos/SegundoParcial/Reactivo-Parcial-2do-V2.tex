\documentclass[12pt,answers]{exam}
\usepackage[top=1.5cm,left=2cm,right=2cm]{geometry}
\usepackage{graphicx}
\graphicspath{{../images/}}
\usepackage{float}
\usepackage[utf8x]{inputenc}
\usepackage{wasysym}
\usepackage{makecell}
\usepackage{tikz}
\usepackage{listings}
        
\usetikzlibrary{shapes,arrows}
\usetikzlibrary{mindmap}
\pagestyle{headandfoot}
%\firstpagefooter{{\bf nota:} Tanto los diagramas de flujo como los programas en c++, deben ser totalmente funcionables y deben ser subidos tanto al classromm(50\%) como al repo de github del equipo de trabajo(50\%).}{}{}



 
% Define block styles
\tikzstyle{decision} = [diamond, draw, fill=blue!20, text width=3.5em, text badly centered, node distance=1.5cm, inner sep=0pt]
\tikzstyle{block} = [rectangle, draw, fill=blue!20, text centered, rounded corners, minimum
    height=1em, minimum width=0.5cm]
\tikzstyle{line} = [draw, -latex']
\tikzstyle{cloud} = [draw, ellipse,fill=red!10, node distance=2cm,
minimum height=2em]
\tikzstyle{conector}=[circle, scale=0.75, color=white, fill=blue!10]

\tikzstyle{output1} = [signal, signal from=nowhere, signal to=east, minimum width=2cm, minimum height=0.5cm, text centered, draw=black, fill=blue!10]

\tikzstyle{input1} = [signal, signal from=east, signal to=nowhere, minimum width=2cm, minimum height=0.5cm, text centered, draw=black, fill=red!30]




\definecolor{codegreen}{rgb}{0,0.6,0}
\definecolor{codegray}{rgb}{0.5,0.5,0.5}
\definecolor{codepurple}{rgb}{0.58,0,0.82}
\definecolor{backcolour}{rgb}{0.95,0.95,0.92}






\begin{document}

\thispagestyle{headandfoot}


\begin{minipage}[H]{0.10\linewidth}
  \flushleft
  \includegraphics[scale=0.12]{images/logoutlvte} 
\end{minipage}
\begin{minipage}[H]{0.70\linewidth}
  \begin{center}
    UNIVERSIDAD TÉCNICA LUIS VARGAS TORRES\\  FACULTAD DE INGENIERÍAS\\
    CARRERA DE TECNOLOGÍAS DE LA INFORMACIÓN \\ Examen del Segundo hemisemestre
  \end{center}
\end{minipage}
\begin{minipage}[H]{0.10\linewidth}
    \flushleft
    \includegraphics[scale=0.3]{images/logofit}
\end{minipage}

\begin{table}[H]
  \centering
  \begin{tabular}[H]{llll}
    Asignatura: & Fundamentos de Programación & Periodo & 2021-1S\\
              &                     &            & \\     
    Apellidos y nombres: &\rule{7cm}{0.4pt}    &  & \\
              &                     &            & \\
    Fecha: &\rule{5cm}{0.4pt}   & Paralelo: & {\Large A} \\
  \end{tabular}
\end{table}




\begin{questions}

\question[25] ¿Cuál sera la salida del siguiente programa.?



\lstinputlisting[
    caption=Example C++,
    label={lst:listing-cpp},
    language=C++,
    backgroundcolor=\color{backcolour},   
    commentstyle=\color{codegreen},
    keywordstyle=\color{magenta},
    numberstyle=\tiny\color{codegray},
    stringstyle=\color{codepurple},
    basicstyle=\ttfamily\footnotesize,
    breakatwhitespace=false,         
    breaklines=true,                 
    keepspaces=true,                 
    numbers=left,       
    numbersep=5pt,                  
    showspaces=false,                
    showstringspaces=false,
    showtabs=false,                  
    tabsize=2,
    ]{programa1.cpp} 

Posibles respuestas:\\

\begin{checkboxes}
\choice   Twelve.
\choice   Twelve Zero.
\choice   Default.
\CorrectChoice   Compilation fails.
\end{checkboxes}
    
  
  
\newpage
\question[25] Indica cuántas iteraciones del bloque se realizan al ejecutar el siguiente bucle.


\lstinputlisting[
    caption=Example C++,
    label={lst:listing-cpp},
    language=C++,
    backgroundcolor=\color{backcolour},   
    commentstyle=\color{codegreen},
    keywordstyle=\color{magenta},
    numberstyle=\tiny\color{codegray},
    stringstyle=\color{codepurple},
    basicstyle=\ttfamily\footnotesize,
    breakatwhitespace=false,         
    breaklines=true,                 
    keepspaces=true,                 
    numbers=left,       
    numbersep=5pt,                  
    showspaces=false,                
    showstringspaces=false,
    showtabs=false,                  
    tabsize=2,
    ]{programa2.cpp} 


Posibles respuestas:\\

\begin{checkboxes}
\choice   36.
\choice   0.
\choice   4.
\CorrectChoice   9.
\choice   Ninguna de las anteriores.
\end{checkboxes}


    
\newpage
\question[25] Indica que valores se escribe cuando se compila y ejecuta el siguiente programa. 


\lstinputlisting[
    caption=Example C++,
    label={lst:listing-cpp},
    language=C++,
    backgroundcolor=\color{backcolour},   
    commentstyle=\color{codegreen},
    keywordstyle=\color{magenta},
    numberstyle=\tiny\color{codegray},
    stringstyle=\color{codepurple},
    basicstyle=\ttfamily\footnotesize,
    breakatwhitespace=false,         
    breaklines=true,                 
    keepspaces=true,                 
    numbers=left,       
    numbersep=5pt,                  
    showspaces=false,                
    showstringspaces=false,
    showtabs=false,                  
    tabsize=2,
    ]{programa3.cpp} 



Posibles respuestas:\\

\begin{checkboxes}
\choice   3,6,9.
\CorrectChoice   3,6,11.
\choice   4,7,11.
\choice 3,6,3.
\choice   Ninguna de las anteriores.
\end{checkboxes}


\newpage
\question[25] Indica que valores se escribe cuando se compila y ejecuta el siguiente programa. 


\lstinputlisting[
    caption=Example C++,
    label={lst:listing-cpp},
    language=C++,
    backgroundcolor=\color{backcolour},   
    commentstyle=\color{codegreen},
    keywordstyle=\color{magenta},
    numberstyle=\tiny\color{codegray},
    stringstyle=\color{codepurple},
    basicstyle=\ttfamily\footnotesize,
    breakatwhitespace=false,         
    breaklines=true,                 
    keepspaces=true,                 
    numbers=left,       
    numbersep=5pt,                  
    showspaces=false,                
    showstringspaces=false,
    showtabs=false,                  
    tabsize=2,
    ]{programa4.cpp} 



Posibles respuestas:\\

\begin{checkboxes}
\choice   3,7,4,4.
\choice   3,7,-4,-4.
\choice   3,7,4,-4.
\choice 3,7,-10,4.
\CorrectChoice   Ninguna de las anteriores.
\end{checkboxes}



\newpage
\question[25] Indica que valores se escribe cuando se compila y ejecuta el siguiente programa. 


\lstinputlisting[
    caption=Example C++,
    label={lst:listing-cpp},
    language=C++,
    backgroundcolor=\color{backcolour},   
    commentstyle=\color{codegreen},
    keywordstyle=\color{magenta},
    numberstyle=\tiny\color{codegray},
    stringstyle=\color{codepurple},
    basicstyle=\ttfamily\footnotesize,
    breakatwhitespace=false,         
    breaklines=true,                 
    keepspaces=true,                 
    numbers=left,       
    numbersep=5pt,                  
    showspaces=false,                
    showstringspaces=false,
    showtabs=false,                  
    tabsize=2,
    ]{programa5.cpp} 




Posibles respuestas:\\

\begin{checkboxes}
\choice   3,7,11.
\choice   4,7,9.
\choice   5,8,13.
\choice   5,8,9.
\CorrectChoice   Ninguna de las anteriores.
\end{checkboxes}


    
\newpage
\question[25] Crear una función suma(float,float) que calcule la suma de dos números y utilizar esta función desde un programa principal main() para calcular 10 números, el resultado de la suma deberá ser presentado por pantalla. 
    


\lstinputlisting[
    caption=Example C++,
    label={lst:listing-cpp},
    language=C++,
    backgroundcolor=\color{backcolour},   
    commentstyle=\color{codegreen},
    keywordstyle=\color{magenta},
    numberstyle=\tiny\color{codegray},
    stringstyle=\color{codepurple},
    basicstyle=\ttfamily\footnotesize,
    breakatwhitespace=false,         
    breaklines=true,                 
    keepspaces=true,                 
    numbers=left,       
    numbersep=5pt,                  
    showspaces=false,                
    showstringspaces=false,
    showtabs=false,                  
    tabsize=2,
    ]{programa6.cpp} 


    \begin{solution}

\lstinputlisting[
    caption=Example C++,
    label={lst:listing-cpp},
    language=C++,
    backgroundcolor=\color{backcolour},   
    commentstyle=\color{codegreen},
    keywordstyle=\color{magenta},
    numberstyle=\tiny\color{codegray},
    stringstyle=\color{codepurple},
    basicstyle=\ttfamily\footnotesize,
    breakatwhitespace=false,         
    breaklines=true,                 
    keepspaces=true,                 
    numbers=left,       
    numbersep=5pt,                  
    showspaces=false,                
    showstringspaces=false,
    showtabs=false,                  
    tabsize=2,
    ]{programa6.cpp} 


      
      \end{solution}



    \newpage
    
\question[25] Implemente un programa que permita ingresar una fecha, lo almacene en la estructura fecha, y los presente por pantalla.
  
\lstinputlisting[
    caption=Example C++,
    label={lst:listing-cpp},
    language=C++,
    backgroundcolor=\color{backcolour},   
    commentstyle=\color{codegreen},
    keywordstyle=\color{magenta},
    numberstyle=\tiny\color{codegray},
    stringstyle=\color{codepurple},
    basicstyle=\ttfamily\footnotesize,
    breakatwhitespace=false,         
    breaklines=true,                 
    keepspaces=true,                 
    numbers=left,       
    numbersep=5pt,                  
    showspaces=false,                
    showstringspaces=false,
    showtabs=false,                  
    tabsize=2,
    ]{programa7.cpp} 



\question[25] Cree un programa que permita ingresar cinco números, realice la suma de los mismos y los presente por pantalla, para los cual solo se utilizará una variable de tipo puntero.


\lstinputlisting[
    caption=Example C++,
    label={lst:listing-cpp},
    language=C++,
    backgroundcolor=\color{backcolour},   
    commentstyle=\color{codegreen},
    keywordstyle=\color{magenta},
    numberstyle=\tiny\color{codegray},
    stringstyle=\color{codepurple},
    basicstyle=\ttfamily\footnotesize,
    breakatwhitespace=false,         
    breaklines=true,                 
    keepspaces=true,                 
    numbers=left,       
    numbersep=5pt,                  
    showspaces=false,                
    showstringspaces=false,
    showtabs=false,                  
    tabsize=2,
    ]{programa8.cpp} 




\question[25] Un punto en el plano, esta dado por las coordenadas x,y, cuando un punto se mueve el valor de estas coordenadas cambian. Cree una función mover() que reciba tres argumentos y retorne la posición, el retorno debe ser de tipo puntero, luego desd la función principal main() presentar por pantalla las coordenadas con las que inicio y con las que termina.

  \lstinputlisting[
    caption=Example C++,
    label={lst:listing-cpp},
    language=C++,
    backgroundcolor=\color{backcolour},   
    commentstyle=\color{codegreen},
    keywordstyle=\color{magenta},
    numberstyle=\tiny\color{codegray},
    stringstyle=\color{codepurple},
    basicstyle=\ttfamily\footnotesize,
    breakatwhitespace=false,         
    breaklines=true,                 
    keepspaces=true,                 
    numbers=left,       
    numbersep=5pt,                  
    showspaces=false,                
    showstringspaces=false,
    showtabs=false,                  
    tabsize=2,
    ]{programa9.cpp} 



\question[25] Cree una clase llamada estudiante, con el atributo Nombres, que contenga una función llamada setNombres(String n) que permita darle un nombre a la persona y otras función llamada presentar(), que muestre el nombre por pantalla, a través del mensaje ``Mi nombre es NN', donde NN sera el contenido del atributo Nombres.
  Crear un objeto de tipo persona, póngale su nombre de pila, e invocando a la función SetNombres() pongale sus nombres completos y luego presentarlo por pantalla.
 
\lstinputlisting[
    caption=Example C++,
    label={lst:listing-cpp},
    language=C++,
    backgroundcolor=\color{backcolour},   
    commentstyle=\color{codegreen},
    keywordstyle=\color{magenta},
    numberstyle=\tiny\color{codegray},
    stringstyle=\color{codepurple},
    basicstyle=\ttfamily\footnotesize,
    breakatwhitespace=false,         
    breaklines=true,                 
    keepspaces=true,                 
    numbers=left,       
    numbersep=5pt,                  
    showspaces=false,                
    showstringspaces=false,
    showtabs=false,                  
    tabsize=2,
    ]{programa10.cpp} 


  
\end{questions}

\newpage
  


  













\end{document}

%%% Local Variables:
%%% mode: latex
%%% TeX-master: t
%%% End:
