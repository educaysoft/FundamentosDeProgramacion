\documentclass[12pt]{exam}
\usepackage[top=1.5cm,left=2cm,right=2cm]{geometry}
\usepackage{graphicx}
\graphicspath{{images/}}
\usepackage{float}
\usepackage[utf8x]{inputenc}
\usepackage{wasysym}
        \usepackage{makecell}
\usepackage{tikz}
\usetikzlibrary{shapes,arrows}
\usetikzlibrary{mindmap}
\pagestyle{headandfoot}
\firstpagefooter{{\bf nota:} Remplazar SUPRIMERAPELLIDO, SUSEGUNDOAPELLIDO, APELLIDO\_NOMBRE \, \, por los suyos propios.}{}{}



 
% Define block styles
\tikzstyle{decision} = [diamond, draw, fill=blue!20, text width=3.5em, text badly centered, node distance=1.5cm, inner sep=0pt]
\tikzstyle{block} = [rectangle, draw, fill=blue!20, text centered, rounded corners, minimum
    height=1em, minimum width=0.5cm]
\tikzstyle{line} = [draw, -latex']
\tikzstyle{cloud} = [draw, ellipse,fill=red!10, node distance=2cm,
minimum height=2em]
\tikzstyle{conector}=[circle, scale=0.75, color=white, fill=blue!10]

\tikzstyle{output1} = [signal, signal from=nowhere, signal to=east, minimum width=2cm, minimum height=0.5cm, text centered, draw=black, fill=blue!10]

\tikzstyle{input1} = [signal, signal from=east, signal to=nowhere, minimum width=2cm, minimum height=0.5cm, text centered, draw=black, fill=red!30]






\begin{document}

\thispagestyle{headandfoot}


\begin{minipage}[H]{0.10\linewidth}
  \flushleft
  \includegraphics[scale=0.12]{logoutlvte}
\end{minipage}
\begin{minipage}[H]{0.70\linewidth}
  \begin{center}
    UNIVERSIDAD TÉCNICA LUIS VARGAS TORRES\\  FACULTAD DE INGENIERÍAS\\
    CARRERA DE TECNOLOGÍAS DE LA INFORMACIÓN \\ Examen del primer hemisemestre
  \end{center}
\end{minipage}
\begin{minipage}[H]{0.10\linewidth}
    \flushleft
    \includegraphics[scale=0.3]{logofit}
\end{minipage}

\begin{table}[H]
  \centering
  \begin{tabular}[H]{llll}
    Asignatura: & Fundamentos de Programación & Periodo & 2021-1S\\
    Docente: & Ing. Stalin Francis &  & \\
              &                     &            & \\     
    Apellidos y nombres: &\rule{7cm}{0.4pt}    &  & \\
              &                     &            & \\
    Fecha: &\rule{5cm}{0.4pt}   & Paralelo:  B& \\
  \end{tabular}
\end{table}



\begin{questions}
\question[20] (Termux) Ubicándose en la carpeta de trabajo (home), escriba los comandos  para crear un directorio
llamado \textbf{EXAMEN}, dentro de este cree un directorio llamado  \textbf{SUPRIMERAPELLIDO} dentro de otro directorio llamado 
\textbf{SUSEGUNDOAPELLIDO} que también debe ser creado, visualizar toda la estructura de directorio creada.\\


\begin{minipage}[H]{0.40\linewidth}
  \noindent \rule{7cm}{0.4pt} \\
  
  \noindent \rule{7cm}{0.4pt} \\
  
  \noindent \rule{7cm}{0.4pt} \\
  
  \noindent \rule{7cm}{0.4pt} 
  
\end{minipage} \hspace{2cm} 
\begin{minipage}[H]{0.40\linewidth}
    \noindent \rule{7cm}{0.4pt} \\
  
    \noindent \rule{7cm}{0.4pt} \\
  
    \noindent \rule{7cm}{0.4pt} \\
  
    \noindent \rule{7cm}{0.4pt} 
  
\end{minipage}

\question[20] (Vim) Ubicandose otra vez en el directorio de trabajo (home), escriba los comandos necesarios que permitan crear un archivo
llamado \textbf{programa.cpp} dentro del directorio
\textbf{SUPRIMERAPELLIDO} creado anteriormente, dentro del
archivo escribir la frase \textbf{``Aqui va el código de un programa en c++''}, finalmente guardar el archivo y salir.\\

\begin{minipage}[H]{0.40\linewidth}
  \noindent \rule{7cm}{0.4pt} \\
  
  \noindent \rule{7cm}{0.4pt} \\
  
  \noindent \rule{7cm}{0.4pt} \\
  
  \noindent \rule{7cm}{0.4pt}
\end{minipage} \hspace{2cm} 
\begin{minipage}[H]{0.40\linewidth}
    \noindent \rule{7cm}{0.4pt} \\
  

  
    \noindent \rule{7cm}{0.4pt} \\
  
    \noindent \rule{7cm}{0.4pt}
\end{minipage}


\question[20] (Termux)Ubíquese en la carpeta de trabajo (home), escriba los
comandos  para  cambiar el nombre del archivo anteriormente creado  a \textbf{APELLIDO}\_\textbf{NOMBRE}\textbf{.txt}, luego moverlo a un
nuevo directorio llamado \textbf{Tareas} el cual debe también
ser creado.\\

\begin{minipage}[H]{0.40\linewidth}
  \noindent \rule{7cm}{0.4pt} \\
  

  
  \noindent \rule{7cm}{0.4pt} \\
  
  \noindent \rule{7cm}{0.4pt}
\end{minipage} \hspace{2cm} 
\begin{minipage}[H]{0.40\linewidth}
    \noindent \rule{7cm}{0.4pt} \\
  
  
    \noindent \rule{7cm}{0.4pt} \\
  
    \noindent \rule{7cm}{0.4pt}
\end{minipage}

\newpage


\question[20] De las siguientes afirmaciones, cual NO es la correcta:.

  \begin{parts}
\part
  Un computador no puede trabajar sin el sistema operativo.
\part
  Todo los recursos del computador son controlados por el sistema operativo.
\part
  Es sistema operativo es comparada con el director de una orquesta.
\part
 Todos los sistemas operativos utilizan GUI.
\part

  No todas las anteriores son correctas.
  
\end{parts}


\question[20] El número 1010 en binario se representa en decimal como.
  \begin{parts}
  \part 1010
  \part 10
  \part 16
  \part Las anteriores respuestas no son correctas.
  \end{parts}
  
  
  













\end{questions}


\end{document}

%%% Local Variables:
%%% mode: latex
%%% TeX-master: t
%%% End:
